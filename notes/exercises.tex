\documentclass[12pt]{article}
\usepackage{fullpage,graphicx,psfrag,amsmath,amsfonts,verbatim}
\usepackage[small,bf]{caption}

\input defs.tex

\bibliographystyle{alpha}

\title{CVXPY Exercises}
\author{Steven Diamond}
\date{TCMM 2014}

\begin{document}
\maketitle

\begin{enumerate}
\item\emph{Hello world.}
Solve the following optimization problem using CVXPY:
\[
\begin{array}{ll} \mbox{minimize} & x^2 - 2\sqrt{y}\\
\mbox{subject to} & 2 \geq e^x \\
& x + y = 5,
\end{array}
\]
where $x,y \in \reals$ are variables.

Find the optimal values of $x$ and $y$.

\item\emph{Non-negative least squares.}
We wish to recover a sparse, non-negative vector $x \in \reals^n$ from measurements $y \in \reals^m$. Our measurement model tells us that
$$
y = Ax + v,
$$
where $A \in \reals^{m \times n}$ is a known matrix and $v \in \reals^m$ is unknown measurement error. The entries of $v$ are drawn IID from the distribution $\mathcal{N}(0, \sigma^2)$.

We can recover a good estimate of $x$ by solving the optimization problem
\[
\begin{array}{ll} \mbox{minimize} & ||Ax - y||^2_2\\
\mbox{subject to} & x \geq 0.
\end{array}
\]
This problem is called non-negative least squares.

The file \verb+nnls.py+ defines $n$, $m$, $A$, $x$, and $y$. Use CVXPY to estimate $x$ from $y$. First try standard regression, \ie, solve
\[
\begin{array}{ll} \mbox{minimize} & ||Ax - y||^2_2.
\end{array}
\]
Use the plotting code in \verb+nnls.py+ to compare the estimated $x$ with the true $x$. Add the constraint $x \geq 0$ and see how it affects the estimate.

How many measurements $m$ are needed for standard regression to find an accurate $x$? How about non-negative least squares?

\item \emph{Minimum fuel optimal control.}
We consider a vehicle moving along a 2D plane. The vehicle's state at time $t$ is described by $x_t \in \reals^4$, where $(x_{t,0}, x_{t,1})$ is the position of the vehicle in two dimensions and $(x_{t,0}, x_{t,1})$ is the vehicle velocity. At each time $t$ a drive force $(u_{t,0}, u_{t,1})$ is applied to the vehicle.

The dynamics of the vehicle's motion is given by the the linear recurrence
\[
x_{t+1} = Ax_t + Bu_t, \quad t=0, \ldots, N-1,
\]
where $A \in \reals^{4 \times 4}$ and $B \in \reals^{4 \times 2}$ are given. We assume that the initial state is zero, \ie, $x_0 = 0$.

The \emph{minimum fuel optimal control problem}
is to choose the drive force $u_0, \ldots, u_{N-1}$ so as to
minimize the total fuel consumed, which is given by
\[
F = \sum_{t=0}^{N-1} f(u_t),
\]
subject to the constraint that $x_N = x_\mathrm{des}$,
where $N$ is the (given) time horizon, and $x_\mathrm{des} \in \reals^4$
is the (given) desired final or target state.
The function $f:\reals^2 \rightarrow \reals$ is the \emph{fuel use map} and gives the amount of fuel used as a function of the drive force.

We will use
\[
f(a) = \|a\|^2_2 + \gamma\|a\|_1.
\]

The file \verb+optimal_control.py+ defines $N$, $A$, $B$, and $x_\mathrm{des}$. Use CVXPY to solve the minimum fuel optimal control problem for $\gamma \in \{0,1,10,100\}$.

Use the plotting code in \verb+optimal_control.py+ to plot $x$ and $u$ for each $\gamma$.

\item\emph{Power grid single commodity flow.}
Recall the definition of a single commodity flow problem from the talk:
\[
\begin{array}{ll} \mbox{minimize} & \sum_{i=1}^n\phi_{i}(f_{i}) + \sum_{j=1}^p\psi_j(s_j), \\
\mbox{subject to} & \text{zero net flow at each node}
\end{array}
\]
where $f_i$ is the flow on edge $i$, $s_j$ is the external source/sink flow into node $j$, and $\phi_i,\psi_j$ are convex cost functions.

We will apply the single commodity flow framework to a power grid. Let nodes $\{1,\ldots,k\}$ be generators. The output at generator $j$ is $s_j$. Each generator has the constraint $0 \leq s_j \leq U_j$ for some maximum output $U_j$ and the cost function $\psi_j(s_j) = s_j^2$.

Nodes $\{k+1,\ldots,p\}$ are consumers. Each consumer $j$ has a fixed load $L_j$, meaning $s_j = L_j$.

Each edge $i$ has the constraint $|f_i| \leq c_i$ for some capacity $c_i$ and the cost function $\phi_i(f_i) = f_i^2$. The edge flow cost represents power loss.

Explicitly, the power grid single commodity flow problem is
\[
\begin{array}{ll} \mbox{minimize} & \sum_{i=1}^nf_{i}^2 + \sum_{j=1}^k s_j^2, \\
\mbox{subject to} & \text{zero net flow at each node} \\
& 0 \leq s_j \leq U_j, \quad j = 1, \ldots, k \\
& s_j = L_j, \quad j = k+1, \ldots, p \\
& |f_i| \leq c_i, \quad i = 1, \ldots, n.
\end{array}
\]

The file \verb+power_grid.py+ defines the power grid graph, the maximum generator outputs $U$, loads $L$, and edge capacities $c$. Complete the classes \verb+Generator+, \verb+Consumer+, and \verb+CapEdge+ and use them to solve the power grid single commodity flow problem.

Use the plotting code in \verb+power_grid.py+ to plot the edge flows in the solution.
% plot total generation cost vs. total load?

\item\emph{Extra} I've included the code for the total variation in-painting example from the talk in \verb+inpainting.py+. Feel free to play around with it if you have time. Try changing \verb+PROB_PIXEL_LOST+ to increase or decrease the number of known pixels.

\end{enumerate}

\end{document}